\documentclass[12pt,spanish,letterpaper,color]{uchile}
\usepackage{ucs}
\usepackage[utf8x]{inputenc}
\usepackage[spanish]{babel}
\PrerenderUnicode{áéíóúÁÉÍÓÚüÜ}
%%%%%%%%%%%%%%%%%%%%%%
%% Para macros propias
\usepackage{xspace}

%move to class?
%%usa punto en vez de coma en las ecuaciones con decimales, tiene que ir despues de babel
\decimalpoint
%%%%%%%%%%%%%%%%%%%%%%
%% Mathematical packages
\usepackage{amsmath}
\usepackage{amsthm}
\usepackage{amssymb}

%move to class?
%%  paquete para setear los margenes, verificar
\usepackage[right=2cm,left=3cm,top=2cm,bottom=2cm,headsep=0cm,footskip=1cm]{geometry}
%\usepackage{anysize}
%\papersize{27.94cm}{21.59cm}
%\marginsize{3.0cm}{2.0cm}{2.0cm}{2.0cm}


%move to class?
%% paquete para insertar links en pdf
%colorlinks=true, resalta los links con colores
\usepackage[pdftex,pdfborderstyle={/S/U/W 0}, 
pdftitle={Memoria}, 
pdfauthor={Felipe Lema}, 
pdfsubject={Emulación}, pdfkeywords={p2p, kaillera, emulación},
pdfcreator={Felipe Lema}]{hyperref}

%%%%%%%%%%%%%%%%%%%%%%
%% Cite Package
%% must go after hyperref
%% debe ir depues de hyperref
\usepackage{cite}

%%%%%%%%%%%%%%%%%%%%%%
%% Numprint Pacakge
%% allows to print number with automatic thousand separator and decimal separator
%% Ejxamples: \numprint{2006.3}-> 2.006,3 \numprint{20009}-> 20.009
%% permite escribir numeros con separador de miles y con separador decimal. 
%% Ejemplos: \numprint{2006.3}-> 2.006,3 \numprint{20009}-> 20.009
\usepackage[autolanguage]{numprint}
\npthousandsep{.}\npdecimalsign{,}

\newtheorem{lemma}{Lema}%we can specify the counter in brackets
\newtheorem{corollary}{Corolario}
\newtheorem{theorem}{Teorema}
%\renewcommand{\bibliographyname}{Bibliografía u otro nombre}
%%%%%%%%%%%%%%%%%%%%%%
%% Listings Package
%% Printing program/source code (C++ code)
%% Imprime código fuente/ de programa (código C++)
\usepackage{listings}
\usepackage{color}
\definecolor{gray97}{gray}{.97}
\definecolor{gray75}{gray}{.75}
\definecolor{gray45}{gray}{.45}

\lstset{ frame=Ltb,
     framerule=0pt,
     aboveskip=0.5cm,
     framextopmargin=3pt,
     framexbottommargin=3pt,
     framexleftmargin=0.4cm,
     framesep=0pt,
     rulesep=.4pt,
     backgroundcolor=\color{gray97},
     rulesepcolor=\color{black},
     %
     stringstyle=\ttfamily,
     showstringspaces = false,
     basicstyle={\ttfamily\small},
     commentstyle=\color{gray45},
     keywordstyle=\bfseries,
     %
     numbers=left,
     numbersep=15pt,
     numberstyle=\tiny,
     numberfirstline = false,
     breaklines=true,
     %
     float,
     language=C++,
     columns=fixed,
     escapeinside={//*}{\^^M}, %labels
     tabsize=2
   }
\renewcommand{\lstlistlistingname}{Códigos}
\renewcommand{\lstlistingname}{Código}

%%%%%%%%%%%%%%%%%%%%%%
%% Algorithms Package
%% Printing pseudocode
%% Imprime pseudo-código
\usepackage{algorithmic}


%%%%%%%%%%%%%%%%%%%%%%
%% clrscode3e Package
%% Typeset pseudo-code "Intro to Algorithms" style
%% Formatea pseudo-código al estilo de "Introducción a Algoritmos"
\usepackage{clrscode3e}

%%%%%%%%%%%%%%%%%%%%%%
%% subfig Package
%% to have subfigures within figures, or subtables within table floats
\usepackage{subfig}

\begin{document}
%%%%%%%%%%%%%%%%%%%%
%macros
%%%%%%%%%%%%%%%%%%%%
\newcommand{\kai}{\textit{Kaillera}\xspace}
\newcommand{\fba}{\texttt{fba}\xspace}


%%%%%%%%%%%%%%%%%%%%
%Definicion de variables de la portada
%%%%%%%%%%%%%%%%%%%%
%%Logo
%\grafica{escudocolor.png}{Logo de la Universidad de Chile}{logo}{0.6}
%% B. Nombre Institucion
\universidad{Universidad de Chile}
\facultad{Facultad de Ciencias Físicas y Matemáticas}
\departamento{Departamento de Ciencias de la Computación}
%% C. Titulo
\title{Sincronización de eventos en juegos multi-usuario distribuidos}
%% D. Proposito titulacion
\trabajoygrado{Memoria para optar al título de Ingeniero Civil en Computación}
%% E. Autor
\author{Felipe Xavier Lema salas}
%% F. Profesores guia, co-guia e integrantes, los 3 primeros son obligatorios
\profguia{Sr. José Piquer Gardner} %profesor guia
\profcoguia{Sr. Nelson Baloian Tataryan} %profesor co-guia
\profint{Sr. Andrés Farías Riquelme} %profesor integrante
%\profinta{Sr. ZZ ZZ ZZ} %profesor integrante 2, generalemente no es necesario
%\profintb{Sr. ZZZ ZZZ ZZZ} %profesor integrante 3, generalmente no es necesario
%% G. Lugar y fecha
\ciudad{Santiago}
\pais{Chile}
\monthpub{Septiembre}
\yearpub{2009}
%por ahora hay que poner el proyecto en mayusculas y si los saltos de linea deben ir a mano
%\proyecto{ESTE TRABAJO HA SIDO FINANCIADO EN PARTE POR EL PROYECTO \\ FONDECYT 12345678 Y POR EL PROYECTO CORFO 2009/8956-4}

%%%%%%%%%%%%%%%%%%%%%
%% Lista de TODOS y FIXMES, no aparece si es que no hay nada que hacer
\listoftodos
\newpage
%% Portada
\maketitle

%% Pagina optativa, ahora creo que ya no va
%\section{Calificaciones}
%\makeeval
% Necesita revision en caso de uso
\begin{preface} 
%%%%%%%%%%%%%%%%%%%%%%%%%%%%%%
%% Resumen
\section{Resumen}
        Este trabajo consiste en el desarrollo de una biblioteca que permita jugar en línea con
emuladores para la plataforma Windows. Se analiza y contrasta la emulación de juegos que no
poseen una vía para jugarlos en línea con la sincronización y consistencia en sistemas
distribuidos.\\

Por un lado, el distribuir la emulación de un hardware requiere una consistencia fuerte,
sacrificando tiempo de reacción, mientras que en los juegos en línea se desea un tiempo mínimo
entre entrada y reacción. En este trabajo se presenta una propuesta que trata de satisfacer ambos
requerimientos en forma equilibrada. Para ello se define un enfoque novedoso con una forma de
abordar la sincronización controlando completamente la entrada de la máquina emulada.\\

A lo largo de la memoria se presentan varias soluciones incrementales que se probaron
con jugadores reales para validar su usabilidad. Desde una primera solución rápida, se pasó a una
en que el estado de una máquina no se mantiene por el tiempo de ejecución del emulador, sino
por el tiempo virtual de la máquina emulada y, en definitiva, se incorpora un retardo para mejorar
la interactividad con el usuario y un protocolo que se adecúa al comportamiento impredecible de
los emuladores al intentar ajustarles la velocidad de emulación. Como última etapa, frente a una
buena consistencia entre los nodos nuevamente se mejoró la interactividad, permitiendo que las
entradas de los jugadores tuviesen más posibilidades de ser inyectadas. Esto entrega una solución
satisfactoria para usuarios jugadores.\\

Finalmente se discuten posibles problemas no abarcados,
maneras de solucionarlos y
detalles de la implementación resultante.



%%%%%%%%%%%%%%%%%%%%%%%%%%%%
%% Dedicatoria: Pagina Optativa
\dedicatoria{A mi Sol}
%% Pagina optativa
%%%%%%%%%%%%%%%%%%%%%%%%%%%%
%% Agradecimientos: Pagina Optativa
\section{Agradecimientos}
Agradezco a mi profesor guía Jo por su apoyo y orientación a lo largo de este trabajo. Nahid
Akbar (a.k.a. Killer Civilian) por sus opiniones y conociemientos con \kai \textit{p2p}.
\texttt{iq\_132} y los usuarios de los foros de FinalBurn Alpha por su ayuda con el código del
emulador. A Joaquín, Sebastián y Haníbal por sus \textit{feedbacks}.\\

%Finalmente, al buscador Google por responderme tan rápido las dudas técnicas para esta memoria.\\
\begin{flushright}
Felipe Lema S.
\end{flushright}
%%%%%%%%%%%%%%%%%%%%%%%%%%%%%%
%% Indices varios
%% Indice General
\tableofcontents
%% Indice de Tablas : Pagina Optativa
%\listoftables
%% Indice de Figuras : Pagina Optativa
\listoffigures

\end{preface}
%%%%%%%%%%%%%%%%%%%%%%%%%%%%%%

%\input{../escritura/introduccion.tex}
%\input{../escritura/antecedentes.tex}
%\input{../escritura/desarrollo_inicial.tex}
%\input{../escritura/sinc_perdidas.tex}
%\input{../escritura/implementacion.tex}
%\input{../escritura/conclusiones.tex}

%%%%%%%%%%%%%%%%%%%%%%%%%%%%%%
\bibliographystyle{abbrv}
\bibliography{biblio}{}
%\begin{additional}
%\section{Ambiente de desarrollo}

%\section{Glosario}
%\section{Material Complementario}
%\end{additional}

\begin{appendix}
\lstset{ frame=Ltb,
     framerule=0pt,
     aboveskip=0.5cm,
     framextopmargin=3pt,
     framexbottommargin=3pt,
     framexleftmargin=0.4cm,
     framesep=0pt,
     rulesep=.4pt,
     backgroundcolor=\color{gray97},
     rulesepcolor=\color{black},
     %
     stringstyle=\ttfamily,
     showstringspaces = false,
     basicstyle={\ttfamily\footnotesize},
     commentstyle=\color{gray45},
     keywordstyle=\bfseries,
     %
     numbers=left,
     numbersep=15pt,
     numberstyle=\tiny,
     numberfirstline = false,
     breaklines=true,
     %
     float,
     language=C++,
     columns=fixed,
     escapeinside={//*}{\^^M}, %labels
     tabsize=2
   }
\section{Código clase \texttt{p2pSync}}
\label{apend:codigo:p2pS}
%\lstinputlisting{../../../fbasrc029704/src/libs/p2pSync/p2pSync.cpp}
\section{Código clase \texttt{MessageManager}}
\label{apend:codigo:MM}
%\lstinputlisting{../../../fbasrc029704/src/libs/p2pSync/MessageManager.cpp}
\section{Código clase \texttt{P2pPlayer}}
\label{apend:codigo:PP}
%\lstinputlisting{../../../fbasrc029704/src/libs/p2pSync/P2pPlayer.cpp}

\end{appendix}
\end{document}
