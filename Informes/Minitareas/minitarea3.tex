\documentclass[10pt,letterpaper]{article}
\usepackage[spanish]{babel}
\usepackage[utf8]{inputenc}
\usepackage{amsmath}
\usepackage{amsfonts}
\usepackage{amssymb}
\usepackage{graphicx}
\usepackage[table]{xcolor}
\begin{document}
%%%%%%%%%%%%%% TITULO %%%%%%%%%%%%%%%%%%%
\begin{center}
\Huge{Minitarea N\textordmasculine 3 Arquitectura de Computadores\\}
\vspace*{1cm}
\large{Profesor: Pablo Guerrero\\}
\large{Integrantes: Eduardo Acha\\}
\large{\today}
\end{center}
%%%%%%%%%%%%%%%%%%%%%%%%%%%%%%%%%%%%%%%%
\section*{P1}
\par \textbf{Explicación:} las operaciones se calculan en paralelo juntando sus resultados en un MUX que a través de los selectores se elige que operación entregar, para que el circuito se síncrono con un reloj en la salida del MUX la conectamos con un Flip-Flop Data de forma que la salida solo cambiara en los pulsos de bajada gracias a que el registro esta conectado con el reloj.
\\
\begin{figure}[h]
\includegraphics[scale=.6]{circuito1.png}
\centering
\caption{Circuito Calc}
\end{figure}

\begin{figure}[h]
\includegraphics[scale=.4]{circuito2.png}
\centering
\caption{Sub-circuito Calc}
\end{figure}

\begin{figure}[h]
\includegraphics[scale=.4]{tabla2.png}
\caption{Tabla Loggin}
\end{figure}





\end{document}
